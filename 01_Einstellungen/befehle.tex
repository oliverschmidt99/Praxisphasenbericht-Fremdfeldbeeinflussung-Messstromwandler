% =============================================================================
% EIGENE BEFEHLE & GLOBALE EINSTELLUNGEN (befehle.tex)
% -----------------------------------------------------------------------------
% Dies ist deine zentrale Steuerdatei für das gesamte Dokument.
% Hier passt du globale Variablen, Schriftarten, Abstände und
% Standardgrößen an.
% =============================================================================

% --- GLOBALE DEFINITIONEN FÜR DIE TITELSEITE ---
\newcommand{\praktikumstitel}{Bachelorarbeit}
\newcommand{\semester}{}
\newcommand{\versuchsnummer}{}
\newcommand{\versuchstitel}{Fremdfeldbeeinflussung auf Messstromwandler in der Niederspannung}
\newcommand{\gruppe}{}
\newcommand{\studiengang}{Studiengang Elektrotechnik}
\newcommand{\vornameStudEins}{}
\newcommand{\nachnameStudEins}{}
\newcommand{\matrikelnummerStudEins}{}
\newcommand{\vornameStudZwei}{Oliver}
\newcommand{\nachnameStudZwei}{Schmidt}
\newcommand{\matrikelnummerStudZwei}{7023462}
\newcommand{\betreuerEins}{Dr. Sandro Günter}
\newcommand{\betreuerZwei}{Dipl.-Ing. Rainer Ludewig}
\newcommand{\betreuerDrei}{Simon Westerbur, B. Eng.}

% --- HINTERGRUNDBILD ---
\newcommand\BackgroundWave{%
    \put(0,0){%
        \parbox[b][\paperheight]{\paperwidth}{%
            \vfill
            \centering
            \vspace{12.0cm}
            \includegraphics[width=\paperwidth]{03_Ressourcen/Logo/hsel-welle-grey}%
            \vfill
        }%
    }%
}


% #############################################################################
% ### ANFANG: GLOBALE LAYOUT- UND SCHRIFTART-EINSTELLUNGEN                  ###
% #############################################################################
%
% Hier kannst du das gesamte Aussehen deines Dokuments anpassen,
% indem du einfach die Werte und Befehle änderst.

% === HAUPTSCHRIFTARTEN ===
% Um die Schriftart zu wechseln, lade das entsprechende Paket in 'praeambel.tex'
% und ändere hier den Befehl.
% Beispiele für Pakete: \usepackage{mathpazo} (Palatino), \usepackage{newtxtext} (Times)
\newcommand{\haupttextSchrift}{\rmfamily}       % \rmfamily (Serif), \sffamily (Sans-Serif)

% === SCHRIFTGRÖSSEN & -STILE ===
% Beispiele: \sffamily (serifenlos), \rmfamily (Serifen), \ttfamily (Schreibmaschine)
%            \bfseries (fett), \itshape (kursiv), \scshape (Kapitälchen)
%            \Huge, \huge, \LARGE, \Large, \large, \normalsize, \small, \footnotesize
\newcommand{\haupttextGroesse}{\normalsize}
\newcommand{\kapitelSchrift}{\sffamily\bfseries\Large}
\newcommand{\sectionSchrift}{\sffamily\bfseries\large}
\newcommand{\subsectionSchrift}{\sffamily\bfseries\normalsize}
\newcommand{\kopfzeilenSchrift}{\sffamily\small}
\newcommand{\fusszeilenSchrift}{\sffamily\itshape\footnotesize}

% === ABSTÄNDE ===
\newcommand{\zeilenabstand}{1.15}               % 1.0 für einfach, 1.15 für etwas luftiger, 1.5 für 1,5-fachen Abstand
\newcommand{\absatzabstand}{0.5em}              % Abstand zwischen Absätzen (z.B. 1em, 0.5cm, etc.)
\newcommand{\absatzeinzug}{0em}                 % Einzug der ersten Zeile eines Absatzes (0em für linksbündig)

% === KOPF- UND FUSSZEILE ===
\newcommand{\kopfzeilenHoehe}{2.5cm}             % Reservierter Platz für die Kopfzeile
\newcommand{\fusszeilenAbstand}{1.5cm}           % Abstand der Fußzeile vom Text

% === STANDARDGRÖSSEN FÜR BILDER ===
% Definiert drei Standardbreiten für Bilder, relativ zur Textbreite.
% Verwendung im Text: \includegraphics[\bildKlein]{pfad/zum/bild.png}
\newcommand{\bildKlein}{width=0.3\textwidth}    % Kleines Bild, ca. 1/3 der Textbreite
\newcommand{\bildMittel}{width=0.6\textwidth}   % Mittleres Bild, ca. 2/3 der Textbreite
\newcommand{\bildGross}{width=0.9\textwidth}    % Großes Bild, fast die gesamte Textbreite

% #############################################################################
% ### ENDE: GLOBALE LAYOUT- UND SCHRIFTART-EINSTELLUNGEN                    ###
% #############################################################################