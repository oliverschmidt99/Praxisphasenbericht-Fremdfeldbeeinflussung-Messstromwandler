% =============================================================================
% EIGENE BEFEHLE & GLOBALE EINSTELLUNGEN (befehle.tex)
% -----------------------------------------------------------------------------
% Dies ist deine zentrale Steuerdatei für das gesamte Dokument.
% Hier passt du globale Variablen, Schriftarten, Abstände und
% Standardgrößen an.
% =============================================================================

% --- GLOBALE DEFINITIONEN FÜR DIE TITELSEITE ---
\newcommand{\praktikumstitel}{Praxisphasenbericht}
\newcommand{\semester}{}
\newcommand{\versuchsnummer}{}
\newcommand{\versuchstitel}{Fremdfeldbeeinflussung auf Messstromwandler in der Niederspannung}
\newcommand{\gruppe}{}
\newcommand{\studiengang}{Studiengang Elektrotechnik}

% --- Studentendefinitionen (wie von dir bereitgestellt) ---
\newcommand{\vornameStudEins}{}
\newcommand{\nachnameStudEins}{}
\newcommand{\matrikelnummerStudEins}{}
\newcommand{\vornameStudZwei}{Oliver}
\newcommand{\nachnameStudZwei}{Schmidt}
\newcommand{\matrikelnummerStudZwei}{7023462}
\newcommand{\autorenname}{\vornameStudZwei~\nachnameStudZwei} % Dieser Befehl ist gut!

% --- Betreuerdefinitionen (wie von dir bereitgestellt) ---
\newcommand{\betreuerEins}{Dr. Sandro Günter}
\newcommand{\betreuerZwei}{Dipl.-Ing. Holger Kuhlemann}
\newcommand{\betreuerDrei}{Dipl.-Ing. Rainer Ludewig}
\newcommand{\betreuerVier}{Simon Westerbur, B. Eng.}

% === NEU: Definitionen für Hochschule und Firma (für 00_deckblatt.tex) ===

% --- Hochschule (Basierend auf deinem Original-Deckblatt) ---
\newcommand{\hochschullogo}{03_Ressourcen/Logo/Technik.png}
\newcommand{\hochschulname}{Hochschule Emden/Leer}
\newcommand{\fachbereich}{Fachbereich Technik}
\newcommand{\abteilung}{Abteilung Elektrotechnik und Informatik}
\newcommand{\hochschuladresse}{Constantiaplatz 4 \\ 26723 Emden} % Beispiel-Adresse, kannst du anpassen

% --- Firma (Basierend auf deiner Eingabe) ---
\newcommand{\firmenlogo}{03_Ressourcen/Logo/logo_rolf-janssen_2024.pdf} % Pfad korrigiert
\newcommand{\firma}{Rolf Janssen GmbH Elektrotechnische Werke}
\newcommand{\firmenadresse}{Emsstraße 4 \\ 26603 Aurich} % Adresse habe ich für dich recherchiert

% =====================================================================


% --- HINTERGRUNDBILD ---
\newcommand\BackgroundWave{%
    \put(0,0){%
        \parbox[b][\paperheight]{\paperwidth}{%
            \vfill
            \centering
            \vspace{12.0cm}
            \includegraphics[width=\paperwidth]{03_Ressourcen/Logo/hsel-welle-grey}%
            \vfill
        }%
    }%
}


% #############################################################################
% ### ANFANG: GLOBALE LAYOUT- UND SCHRIFTART-EINSTELLUNGEN                ###
% #############################################################################
%
% Hier kannst du das gesamte Aussehen deines Dokuments anpassen,
% indem du einfach die Werte und Befehle änderst.
% === HAUPTSCHRIFTARTEN ===
% Um die Schriftart zu wechseln, lade das entsprechende Paket in 'praeambel.tex'
% und ändere hier den Befehl.
% Beispiele für Pakete: \usepackage{mathpazo} (Palatino), \usepackage{newtxtext} (Times)
\newcommand{\haupttextSchrift}{\rmfamily}       % \rmfamily (Serif), \sffamily (Sans-Serif)

% === SCHRIFTGRÖSSEN & -STILE ===
% Beispiele: \sffamily (serifenlos), \rmfamily (Serifen), \ttfamily (Schreibmaschine)
%           \bfseries (fett), \itshape (kursiv), \scshape (Kapitälchen)
%           \Huge, \huge, \LARGE, \Large, \large, \normalsize, \small, \footnotesize
\newcommand{\haupttextGroesse}{\normalsize}
\newcommand{\kapitelSchrift}{\sffamily\bfseries\Large}
\newcommand{\sectionSchrift}{\sffamily\bfseries\large}
\newcommand{\subsectionSchrift}{\sffamily\bfseries\normalsize}
\newcommand{\kopfzeilenSchrift}{\haupttextSchrift\small}
\newcommand{\fusszeilenSchrift}{\haupttextSchrift\footnotesize}

% === ABSTÄNDE ===
\newcommand{\zeilenabstand}{1.15}            % 1.0 für einfach, 1.15 für etwas luftiger, 1.5 für 1,5-fachen Abstand
\newcommand{\absatzabstand}{0.5em}             % Abstand zwischen Absätzen (z.B. 1em, 0.5cm, etc.)
\newcommand{\absatzeinzug}{0em}              % Einzug der ersten Zeile eines Absatzes (0em für linksbündig)

% === KOPF- UND FUSSZEILE ===
\newcommand{\kopfzeilenHoehe}{2.5cm}           % Reservierter Platz für die Kopfzeile
\newcommand{\fusszeilenAbstand}{1.5cm}           % Abstand der Fußzeile vom Text


% =============================================================================
% ### BEFEHLE FÜR AUTOMATISIERTE UMGEBUNGEN ###
% =============================================================================

% Befehl für eine kleine Abbildung (ca. 1/3 der Seitenbreite)
% #1 = Dateipfad
% #2 = Caption (Unterschrift)
% #3 = Label zur Referenzierung
\newcommand{\einfuegenBildKlein}[3]{%
    \begin{figure}[H]
        \centering
        \includegraphics[width=0.4\textwidth]{#1'Bild'}
        \caption{#2}
        \label{#3}
    \end{figure}%
}

% Befehl für eine mittlere Abbildung (ca. 2/3 der Seitenbreite)
% #1 = Dateipfad
% #2 = Caption (Unterschrift)
% #3 = Label zur Referenzierung
\newcommand{\einfuegenBildMittel}[3]{%
    \begin{figure}[H]
        \centering
        \includegraphics[width=0.6\textwidth]{#1}
        \caption{#2}
        \label{#3}
    \end{figure}%
}

% Befehl für eine große Abbildung (fast die ganze Seitenbreite)
% #1 = Dateipfad
% #2 = Caption (Unterschrift)
% #3 = Label zur Referenzierung
\newcommand{\einfuegenBildGross}[3]{%
    \begin{figure}[H]
        \centering
        \includegraphics[width=0.9\textwidth]{#1}
        \caption{#2}
        \label{#3}
    \end{figure}%
}

% #############################################################################
% ### ENDE: GLOBALE LAYOUT- UND SCHRIFTART-EINSTELLUNGEN                ###
% #############################################################################