% =============================================================================
% PRÄAMBEL (praeambel.tex)
% -----------------------------------------------------------------------------
% Lädt alle Pakete und globalen Einstellungen.
% =============================================================================

% --- GRUNDEINSTELLUNGEN ---
\usepackage[utf8]{inputenc}
\usepackage[T1]{fontenc}
\usepackage[ngerman]{babel}
\usepackage{graphicx}
\usepackage{csquotes}

% --- LITERATURVERZEICHNIS (BIBLATEX) ---
\usepackage[backend=biber, style=alphabetic, sorting=ynt]{biblatex}

% --- MATHEMATIK ---
\usepackage{amsmath, amssymb, amsthm, latexsym, derivative}

% --- GRAFIKEN & DIAGRAMME ---
\usepackage{epsfig, tikz, subcaption, caption, adjustbox, eso-pic}
\usepackage[european]{circuitikz}
\usepackage{siunitx}
\sisetup{locale = DE}

\usepackage{svg}
\usepackage{filecontents}

% --- TABELLEN ---
\usepackage[table]{xcolor} % table-Option für colortbl-Kompatibilität
\usepackage{booktabs, multirow, colortbl}

% --- SONSTIGE PAKETE ---
\usepackage{enumitem}
\usepackage{makeidx}
% \makeindex % Vorerst auskommentiert
\usepackage{totpages} % Robustes Paket für Gesamtseitenzahl
\usepackage[colorlinks=true, linkcolor=blue, citecolor=green, urlcolor=magenta, hidelinks]{hyperref}
\usepackage[acronym, nonumberlist, toc, section]{glossaries}
\makeglossaries
\usepackage{chngcntr}
\usepackage[normalem]{ulem}
\usepackage{calc}
\usepackage{pdfpages}
\usepackage{blindtext}
\usepackage{float}
\usepackage{newfloat}
\usepackage[most]{tcolorbox}
\usepackage[framemethod=TikZ]{mdframed}
\usepackage{etoolbox}

% --- LAYOUT-DATEI LADEN ---


% --- BENUTZERDEFINIERTE EINSTELLUNGEN ---
\setlength{\parindent}{0em}
\setlength{\parskip}{0.5em}

% --- DEINE FLOAT-UMGEBUNGEN ---
\DeclareFloatingEnvironment[listname={Diagrammverzeichnis}, name=Diagramm, fileext=lod, placement=htp]{diagram}
\DeclareFloatingEnvironment[listname={Oszillogrammverzeichnis}, name=Oszillogramm, fileext=loo, placement=htp]{oszillo}

% --- ZÄHLER-EINSTELLUNGEN ---
\counterwithin{figure}{section}
\counterwithin{table}{section}
\counterwithin{oszillo}{section}
\counterwithin{diagram}{section}

% --- DEINE BOX-DEFINITIONEN ---
\newmdenv[linecolor=red, linewidth=1pt, frametitle=Hinweis, frametitlebackgroundcolor=red, frametitlefont=\color{white}\bfseries, frametitlerule=true, backgroundcolor=white, roundcorner=2pt]{hinweisbox}
\newmdenv[linecolor=gray, linewidth=1pt, frametitle=Aufgabe, frametitlebackgroundcolor=white, frametitlefont=\color{black}\bfseries, frametitlerule=true, backgroundcolor=white, roundcorner=2pt]{aufgabenbox}
\newcounter{rechnungbox}[section]
\renewcommand{\therechnungbox}{\arabic{rechnungbox}}
\newtcolorbox{rechnungbox}[1][]{colframe=darkgray, colback=white, coltitle=white, colbacktitle=gray, title={Rechnung \therechnungbox}, fonttitle=\bfseries, boxed title style={size=small},
    boxrule=1pt, sharp corners=south, before upper=\stepcounter{rechnungbox}, #1}

% --- DEINE TOGGLES ---
\newtoggle{showtoc}
\newtoggle{showtables}
\newtoggle{showfigures}
\newtoggle{showoszillos}
\newtoggle{showdiagrams}
\toggletrue{showtoc}
\togglefalse{showtables}
\toggletrue{showfigures}
\togglefalse{showoszillos}
\togglefalse{showdiagrams}

% --- Globale Schriftart für den Haupttext aktivieren ---
\AtBeginDocument{\haupttextSchrift}