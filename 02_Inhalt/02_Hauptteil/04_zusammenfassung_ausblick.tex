\section{Zusammenfassung und Ausblick}
\label{chap:zusammenfassung_ausblick}

Diese Arbeit analysierte die Fremdfeldbeeinflussung von Messstromwandlern in Niederspannungsschaltanlagen, die durch die kompakte Bauweise und hohe Ströme benachbarter Leiter entsteht.
Es wurde erfolgreich eine Methodik zur Untersuchung dieser Störeinflüsse entwickelt, die zwei Kernkomponenten kombiniert: eine maßgeschneiderte, \gls{quasi-stationaer}e Simulationsumgebung auf Basis von \gls{femm} und ein systematisches Messverfahren am \gls{erwaermungspruefstand}.
Für die strukturierte Datenerfassung wurde ein VBA-basiertes Messprotokoll-Tool realisiert.


\subsection*{Ausblick}

Das validierte Simulationsmodell bietet eine solide Grundlage für zukünftige Optimierungen.
Als nächste Schritte könnten die Untersuchungen auf ein 3D-FEM-Modell erweitert werden, um auch Streufelder und komplexe Geometrien noch exakter abzubilden.
Basierend auf den gewonnenen Erkenntnissen kann die Simulationsumgebung nun genutzt werden, um konkrete Konstruktionsrichtlinien und Mindestabstände für die Anordnung von Messstromwandlern in zukünftigen Schaltanlagen-Designs abzuleiten.
Dies ermöglicht es der Rolf Janssen GmbH, die Messgenauigkeit bereits in der Planungsphase sicherzustellen und kostspielige Messfehler im Feld zu vermeiden.