\section{Zusammenfassung und Ausblick}
\label{chap:zusammenfassung_ausblick}

Diese Arbeit analysierte die \gls{fremdfeld}beeinflussung von \glspl{messstromwandler} in \glspl{niederspannungsschaltanlage}, die durch die kompakte Bauweise und hohe Ströme benachbarter \glspl{leiter} entsteht.

Es wurde erfolgreich eine Methodik zur Untersuchung dieser Störeinflüsse entwickelt, die zwei Kernkomponenten kombiniert: eine maßgeschneiderte, \gls{quasi-stationaer}e Simulationsumgebung auf Basis von \gls{femm} und ein systematisches Messverfahren am \gls{erwaermungspruefstand}. Für die strukturierte Datenerfassung wurde ein VBA-basiertes Messprotokoll-Tool realisiert.

Die Ergebnisse (Abschnitt \ref{sec:ergebnisse}) belegen, dass die \gls{fremdfeld}kopplung, insbesondere auf die mittlere Phase \acrshort{l2}, zu einer \gls{magnetische-saettigung} des Wandlerkerns führen kann. Dies beeinträchtigt die Messgenauigkeit, vor allem in den unteren Messbereichen (z.B. 5~\% \acrshort{in}), und kann zur Verletzung der \gls{genauigkeitsklasse} führen. Es wurde nachgewiesen, dass eine geometrische Anpassung der \gls{sammelschiene}nanordnung, wie die Verlagerung der Phase \acrshort{l2}, die Störeinkopplung effektiv minimiert.

\subsection*{Ausblick}

Das validierte Simulationsmodell bietet eine solide Grundlage für zukünftige Optimierungen. Als nächste Schritte könnten die Untersuchungen auf ein 3D-FEM-Modell erweitert werden, um auch Streufelder und komplexe Geometrien noch exakter abzubilden.

Basierend auf den gewonnenen Erkenntnissen kann die Simulationsumgebung nun genutzt werden, um konkrete Konstruktionsrichtlinien und Mindestabstände für die Anordnung von \glspl{messstromwandler} in zukünftigen Schaltanlagen-Designs abzuleiten. Dies ermöglicht es der Rolf Janssen GmbH, die Messgenauigkeit bereits in der Planungsphase sicherzustellen und kostspielige Messfehler im Feld zu vermeiden.