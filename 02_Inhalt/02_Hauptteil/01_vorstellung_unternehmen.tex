% =============================================================================
% KAPITEL 1: VORSTELLUNG DES UNTERNEHMENS
% -----------------------------------------------------------------------------
% Hier stellst du das Unternehmen vor, bei dem du die Praxisphase absolviert hast.
% =============================================================================

\section{Vorstellung des Unternehmens}\label{chap:vorstellung_unternehmen}

Das Unternehmen, die Rolf Janssen GmbH, wurde im Jahr 1949 gegründet und blickt auf eine mehr als 75-jährige Geschichte im Bereich der Elektrotechnik zurück. Mit Hauptsitz in Aurich, hat sich das Unternehmen kontinuierlich entwickelt und beschäftigt heute über 200 Mitarbeiter an insgesamt fünf Standorten in Deutschland.

\einfuegenBildGross{03_Ressourcen/Bilder/abbildung_01_firmenzentrale.jpg} {Firmenzentrale der Rolf Janssen GmbH in Aurich} {fig:firmenzentrale}

Das Kerngeschäft der Rolf Janssen GmbH liegt in der Planung, Fertigung und Inbetriebnahme von komplexen elektrotechnischen und leittechnischen Anlagen. Das Leistungsportfolio ist dabei breit diversifiziert und konzentriert sich auf die folgenden Hauptgeschäftsfelder:

\begin{itemize} 
    \item \textbf{Niederspannungsschaltanlagen}: Die Entwicklung und Fertigung von maßgeschneiderten Schaltanlagen bilden eine zentrale Säule des Unternehmens. Hierbei werden individuelle Kundenanforderungen präzise umgesetzt. 
    \item \textbf{Prozessautomatisierung}: Ein weiterer Schwerpunkt ist die Konzeption und Implementierung von modernen Automatisierungslösungen für industrielle Fertigungs- und Verfahrensprozesse. 
    \item \textbf{Diesel-Netzersatzanlagen}: Das Unternehmen stellt Notstromaggregate bereit, die eine unterbrechungsfreie Stromversorgung (USV) für kritische Infrastrukturen und industrielle Anwendungen gewährleisten. 
    \item \textbf{Maritime Lösungen}: Für den anspruchsvollen maritimen Sektor werden spezialisierte Elektro- und Leittechniksysteme entwickelt, die den besonderen Anforderungen auf See (z. B. auf Schiffen oder Offshore-Plattformen) gerecht werden. 
\end{itemize}

Ein wesentliches Alleinstellungsmerkmal der Rolf Janssen GmbH ist die hohe Fertigungstiefe. Diese ermöglicht es, den gesamten Produktionsprozess – vom Rohmaterial bis zum betriebsbereiten Endprodukt – im eigenen Haus zu steuern. Dadurch können Kundenanforderungen nicht nur flexibel, sondern auch kurzfristig umgesetzt werden. Das Unternehmen legt besonderen Wert auf die Realisierung von anspruchsvollen Sonderlösungen und bietet eine durchgängige Projektbetreuung, die von der ersten Angebotsphase über die Konstruktion und Fertigung bis zur finalen Inbetriebnahme reicht. Diese umfassende Herangehensweise positioniert die Rolf Janssen GmbH als kompetenten und zuverlässigen Partner für elektrotechnische Gesamtpakete in diversen Branchen.

