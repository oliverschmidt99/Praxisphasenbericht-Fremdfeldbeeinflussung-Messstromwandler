% =============================================================================
% KAPITEL 1: VORSTELLUNG DES UNTERNEHMENS
% -----------------------------------------------------------------------------
% Hier stellst du das Unternehmen vor, bei dem du die Praxisphase absolviert hast.
% =============================================================================

\section{Vorstellung des Unternehmens}
\label{chap:vorstellung_unternehmen}

Gegründet im Jahr 1949, verfügt das Unternehmen über mehr als 75 Jahre Erfahrung in der Elektrotechnik. Heute beschäftigt es an fünf Standorten in Deutschland über 200 Mitarbeiter.\\


\einfuegenBildGross{03_Ressourcen/Bilder/abbildung_01_firmenzentrale.jpg}
{Firmenzentrale der Rolf Janssen GmbH in Aurich}
{fig:firmenzentrale}


Das Kerngeschäft der Rolf Janssen GmbH umfasst die Planung, Fertigung und Inbetriebnahme von elektrotechnischen und leittechnischen Anlagen. Das Leistungsportfolio ist breit gefächert und lässt sich in folgende Hauptbereiche gliedern:\\

\begin{itemize}
    \item Niederspannungsschaltanlagen: Entwicklung und Herstellung von maßgeschneiderten Schaltanlagen.
    \item Prozessautomatisierung: Konzeption und Implementierung von Automatisierungslösungen für industrielle Prozesse.
    \item Diesel-Netzersatzanlagen: Bereitstellung von Notstromaggregaten zur Sicherstellung einer unterbrechungsfreien Stromversorgung.
    \item Maritime Lösungen: Spezialisierte Elektro- und Leittechnik für den Einsatz auf See.\\
\end{itemize}

Ein wesentliches Merkmal des Unternehmens ist die hohe Fertigungstiefe, die eine flexible und schnelle Umsetzung von Kundenanforderungen vom Rohmaterial bis zum fertigen Produkt ermöglicht. Die Rolf Janssen GmbH legt besonderen Wert auf die Realisierung individueller Sonderlösungen und bietet eine durchgängige Projektbetreuung von der Angebotsphase bis zur finalen Inbetriebnahme. Dies positioniert das Unternehmen als kompetenten Partner für komplexe elektrotechnische Gesamtpakete in diversen Branchen.\\

