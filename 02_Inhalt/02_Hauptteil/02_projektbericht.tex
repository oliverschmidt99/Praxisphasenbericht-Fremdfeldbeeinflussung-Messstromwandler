% =============================================================================
% KAPITEL 2: PROJEKTBERICHT
% -----------------------------------------------------------------------------
% Das Kernstück deines Berichts. Hier beschreibst du dein Projekt.
% =============================================================================

\section{Analyse der Fremdfeldbeeinflussung von Messstromwandlern}
\label{chap:projektbericht}

Im Rahmen dieser Praxisphase wird die \gls{fremdfeld}beeinflussung auf \glspl{messstromwandler} analysiert. Ziel ist es, die physikalischen Ursachen der Messabweichungen zu untersuchen und Lösungsansätze zur Minimierung dieser Störeinflüsse zu entwickeln.

\subsection{Motivation}
\label{sec:motivation}
In modernen Niederspannungsschaltanlagen treffen konstruktive Anforderungen an eine hohe Leistungsdichte auf hohe Betriebsströme. Die daraus resultierende kompakte Anordnung der \glspl{sammelschiene} führt dazu, dass die starken magnetischen Wechselfelder der stromführenden Leiter auf benachbarte \glspl{messstromwandler} koppeln. Diese \gls{fremdfeld}beeinflussung ist eine signifikante Ursache für Messfehler. Besonders kritisch ist diese Kopplung für den Messwandler des mittleren Leiters (\gls{l2}), da dieser von den Feldern der beiden äußeren Phasen (\gls{l1} und \gls{l3}) umschlossen wird.

\subsection{Vorgehen}
\label{sec:vorgehen}
Wie bist du methodisch vorgegangen? Welche Schritte hast du unternommen?

\subsection{Ergebnisse}
\label{sec:ergebnisse}
Was sind die konkreten Ergebnisse deiner Arbeit?