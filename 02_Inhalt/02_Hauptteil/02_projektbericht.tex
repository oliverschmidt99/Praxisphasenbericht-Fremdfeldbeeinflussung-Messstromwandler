% =============================================================================
% KAPITEL 2: PROJEKTBERICHT
% -----------------------------------------------------------------------------
% Das Kernstück deines Berichts. Hier beschreibst du dein Projekt.
% HINWEIS: Die \gls- und \acr-Befehle erzeugen automatisch die Links im PDF.
% =============================================================================

\section{Analyse der Fremdfeldbeeinflussung von Messstromwandlern}
\label{chap:projektbericht}

Im Rahmen dieser Praxisphase wird die \gls{fremdfeld}beeinflussung auf \glspl{messstromwandler} im Bereich der \gls{niederspannung} analysiert. Ziel ist es, die physikalischen Ursachen für Messabweichungen, basierend auf dem \gls{induktionsgesetz}, zu untersuchen und Lösungsansätze zur Minimierung dieser Störeinflüsse zu entwickeln.

\subsection{Motivation}
\label{sec:motivation}

In modernen \Glspl{niederspannungsschaltanlage} treffen konstruktive Anforderungen an eine kompakte Bauweise auf hohe Betriebsströme. Die daraus resultierende kompakte Anordnung der \glspl{sammelschiene} führt dazu, dass die starken magnetischen Wechselfelder der stromführenden \glspl{leiter} auf benachbarte \glspl{messstromwandler} koppeln. Diese \gls{fremdfeld}beeinflussung ist eine signifikante Ursache für Messfehler, welche die Einhaltung der \gls{genauigkeitsklasse} beeinträchtigen kann und somit für Verrechnungszwecke nicht mehr geeignet ist. Besonders kritisch ist diese Kopplung für den Wandler der mittleren Phase \acrshort{l2}, da dieser von den Feldern der beiden äußeren Phasen \acrshort{l1} und \acrshort{l3} beeinflusst wird.

\subsection{Vorgehen}
\label{sec:vorgehen}

Um die gegenseitige Beeinflussung der Magnetfelder zu untersuchen, wurde die Open-Source-Software \gls{femm} eingesetzt. Mit dieser lassen sich verschiedene elektrotechnische Problemstellungen wie magnetische, elektrostatische, stromfluss oder thermische Phänomene simulieren. Da die Software jedoch ausschließlich \glspl{stationaer} Probleme in zwei Dimensionen berechnen kann, wurde für die Analyse der \gls{magnetfeldeinkopplung} in die \glspl{messstromwandler} eine ergänzende Software zur Simulation dynamischer Vorgänge entwickelt.

Die messtechnische Untersuchung der \glspl{messstromwandler} findet im \gls{erwaermungspruefstand} statt. Sie gliedert sich in zwei Bereiche: Zuerst werden die \glspl{interne-parameter} mittels \gls{acv-verfahren}- und \gls{dcv-verfahren}-Messungen ermittelt. Anschließend erfolgt die \gls{genauigkeitsmessung} unter Betriebsbedingungen mit den jeweiligen \glspl{nennstrom} der Wandler.

\subsubsection{Simulation}
\label{sec:vorgehen:simulation}



\subsubsection{Messung}
\label{sec:vorgehen:messung}
% Hier Inhalt einfügen

\subsection{Ergebnisse}
\label{sec:ergebnisse}
% Hier Inhalt einfügen