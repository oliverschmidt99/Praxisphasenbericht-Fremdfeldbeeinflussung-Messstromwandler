% =============================================================================
% KAPITEL 2: PROJEKTBERICHT
% -----------------------------------------------------------------------------
% Das Kernstück deines Berichts. Hier beschreibst du dein Projekt.
% HINWEIS: Die \gls- und \acr-Befehle erzeugen automatisch die Links im PDF.
% =============================================================================

\section{Analyse der Fremdfeldbeeinflussung von Messstromwandlern}
\label{chap:projektbericht}

Im Rahmen dieser Praxisphase wird die \gls{fremdfeld}beeinflussung auf \glspl{messstromwandler} im Bereich der \gls{niederspannung} analysiert. Ziel ist es, die physikalischen Ursachen für Messabweichungen, basierend auf dem \gls{induktionsgesetz}, zu untersuchen und Lösungsansätze zur Minimierung dieser Störeinflüsse zu entwickeln.

\subsection{Motivation}
\label{sec:motivation}

In modernen \Glspl{niederspannungsschaltanlage} treffen konstruktive Anforderungen an eine kompakte Bauweise auf hohe Betriebsströme. Die daraus resultierende kompakte Anordnung der \glspl{sammelschiene} führt dazu, dass die starken magnetischen Wechselfelder der stromführenden \glspl{leiter} auf benachbarte \glspl{messstromwandler} koppeln. Diese \gls{fremdfeld}beeinflussung ist eine signifikante Ursache für Messfehler, welche die Einhaltung der \gls{genauigkeitsklasse} beeinträchtigen kann und somit für Verrechnungszwecke nicht mehr geeignet ist. Besonders kritisch ist diese Kopplung für den Wandler der mittleren Phase \acrshort{l2}, da dieser von den Feldern der beiden äußeren Phasen \acrshort{l1} und \acrshort{l3} beeinflusst wird.

\subsection{Vorgehen}
\label{sec:vorgehen}

Um die gegenseitige Beeinflussung der Magnetfelder zu untersuchen, wurde die Open-Source-Software \gls{femm} eingesetzt. Mit dieser lassen sich verschiedene elektrotechnische Problemstellungen wie magnetische, elektrostatische, stromfluss oder thermische Phänomene simulieren. Da die Software jedoch ausschließlich \glspl{stationaer} Probleme in zwei Dimensionen berechnen kann, wurde für die Analyse der \gls{magnetfeldeinkopplung} in die \glspl{messstromwandler} eine ergänzende Software zur Simulation dynamischer Vorgänge entwickelt.

Die messtechnische Untersuchung der \glspl{messstromwandler} findet im \gls{erwaermungspruefstand} statt. Sie gliedert sich in zwei Bereiche: Zuerst werden die \glspl{interne-parameter} mittels \gls{acv-verfahren}- und \gls{dcv-verfahren}-Messungen ermittelt. Anschließend erfolgt die \gls{genauigkeitsmessung} unter Betriebsbedingungen mit den jeweiligen \glspl{nennstrom} der Wandler.

\subsubsection{Vorbereitung}
\label{sec:vorgehen:vorbereitung}

Zur Vorbereitung der messtechnischen und simulativen Untersuchungen wurden im Vorfeld \glspl{messstromwandler} verschiedener Hersteller mit unterschiedlichen \glspl{nennstrom} beschafft.

Parallel dazu erfolgte eine umfassende Einarbeitung in die theoretischen Grundlagen der Messwandlertechnik und der \gls{fremdfeld}problematik. Dies umfasste die Recherche und Analyse relevanter wissenschaftlicher Artikel (Paper) und Fachbücher. Ergänzend wurden erste analytische Berechnungen zur \gls{magnetfeldeinkopplung} mittels \gls{matlab}  durchgeführt.

\subsubsection{Simulation}
\label{sec:vorgehen:simulation}

Für die Simulation wurden verschiedene Ansätze verfolgt. Zunächst kam ein in Matlab implementiertes Skript zum Einsatz, das die Magnetfelder auf Grundlage des Biot-Savart-Gesetzes berechnet. Schnell zeigte sich jedoch, dass dieser Ansatz die physikalische Realität nur unzureichend abbildet. Daher wurde eine alternative Methode erforderlich.

Nach einer ausführlichen Recherche geeigneter, kostenfreier Open-Source-Software für Magnetfeldsimulationen fiel die Wahl auf FEMM (Finite Element Method Magnetics).

FEMM ist ein Programm, das magnetische Probleme mithilfe der Finite-Elemente-Methode (FEM) löst. Der entscheidende Vorteil gegenüber dem Biot-Savart-Ansatz liegt darin, dass FEMM nicht nur komplexe Leitergeometrien, sondern auch die nichtlinearen Materialeigenschaften ferromagnetischer Werkstoffe (z. B. B(H)-Kennlinie und magnetische Sättigung) berücksichtigt. Genau diese Eigenschaften wurden beim reinen Biot-Savart-Modell vernachlässigt, was zu deutlichen Abweichungen führte. Daher wurde das Modell in FEMM neu aufgebaut, um die Sättigungseffekte realitätsnah abzubilden.

\textbf{FEMM (Finite Element Method Magnetics)}

FEMM ist eine Programmsuite zur Lösung elektromagnetischer Probleme in zweidimensionalen planaren oder achsensymmetrischen Geometrien. Das Programm kann aktuell linear und nichtlinear magnetostatische, zeitharmonische magnetische, elektrostatische sowie stationäre Wärmeflussprobleme berechnen.

Für die spezifischen Anforderungen des Messaufbaus wurde auf Basis von FEMM eine maßgeschneiderte Softwarelösung entwickelt, die die Fremdfeldbeeinflussung gezielt abbildet.
Diese Lösung stellt eine Weboberfläche bereit, über die sich die Simulationen definieren, parametrieren und starten lassen. Die Oberfläche dient als einheitliche Vorlage (Template) und ermöglicht die standardisierte Erstellung von Simulationsmodellen für verschiedene Wandlertypen und Sammelschienenanordnungen.

Da FEMM ausschließlich stationäre Fälle betrachtet, erweitert die Hilfssoftware diesen Ansatz zu einem quasi-stationären Modell.
Die Hauptschwierigkeit bestand darin, die zeitharmonischen 50-Hz-Wechselfelder der \glspl{sammelschiene} darzustellen, obwohl FEMM primär für magnetostatische (DC-)Probleme ausgelegt ist. Um dennoch dynamische Vorgänge zu simulieren, wurde ein quasi-stationärer Ansatz implementiert:
Zu diskreten Zeitpunkten $t_n$ innerhalb einer Netzperiode berechnet die Hilfssoftware die jeweiligen Momentanwerte der Phasenströme $i_{L1}(t_n)$, $i_{L2}(t_n)$ und $i_{L3}(t_n)$. Diese Werte werden automatisiert an FEMM übergeben, das für jeden Zeitschritt ein stationäres Magnetfeldproblem löst. Durch die Aneinanderreihung der Einzelergebnisse – insbesondere des magnetischen Flusses $\Psi(t_n)$ im Wandlerkern – lässt sich der zeitliche Verlauf über eine volle Periode rekonstruieren.

Der detaillierte Modellaufbau in FEMM erfolgte in einem zweidimensionalen Querschnitt, der die Anordnung der \glspl{sammelschiene} und des \gls{messstromwandler} abbildet:

\begin{itemize}
    \item \textbf{Geometrie:} Die drei Phasenleiter wurden entsprechend ihrer realen Querschnittsabmessungen (z. B. 100 mm × 10 mm) und ihres Mittenabstands modelliert. Ebenso wurde der Ringkern des \gls{messstromwandler} mit seinem exakten Innen- und Außendurchmesser sowie seiner Position relativ zu den Leitern nachgebildet.
    \item \textbf{Materialien:} Den Leitern wurde Kupfer mit seiner spezifischen elektrischen Leitfähigkeit zugewiesen, der Umgebungsraum als Luft ($\mu_r = 1$) definiert. Der ferromagnetische Wandlerkern erhielt eine nichtlineare B(H)-Kennlinie aus der FEMM-Materialbibliothek (z. B. M-19 Steel), um das Sättigungsverhalten realitätsnah zu erfassen.
\end{itemize}


\einfuegenBildGross{03_Ressourcen/Bilder/website_Simulations-Vorschau.png}{Weboberfläche der Simulationssoftware mit Vorschau des Modells}{fig:weboberflaeche}


\einfuegenBildGross{03_Ressourcen/Bilder/website_Simulations-einstellungen.png}{Weboberfläche der Simulationssoftware mit Einstellmöglichkeiten für die Simulation}{fig:weboberflaeche_einstellungen}





\subsubsection{Messung}
\label{sec:vorgehen:messung}
% Hier Inhalt einfügen

Im weitern Verlauf mussten die Wandler einer Realenmessung unterzogen werden.


\subsection{Ergebnisse}
\label{sec:ergebnisse}
% Hier Inhalt einfügen

In der Abbildung \ref{fig:simulation_aufbau} kann man den groben Aufbau der Simulation betrachten, welcher sich in drei Kategorien gliedert einmal in erstellung der Bauteile, erstellung der Simulation und die Visuelle darstellung der Ergebinsse aus der Simulation.


\einfuegenBildGross{04_Zeichnungen/verlauf_femm.png}{Aufbau der Simulation}{fig:simulation_aufbau}