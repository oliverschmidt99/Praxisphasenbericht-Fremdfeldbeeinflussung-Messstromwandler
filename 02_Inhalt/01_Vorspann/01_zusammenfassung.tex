% =============================================================================
% ABSTRACT (01_abstract.tex)
% -----------------------------------------------------------------------------
% Eine kurze, prägnante Zusammenfassung deiner gesamten Arbeit.
% (Thema, Vorgehen/Methode, wichtigste Ergebnisse, Schlussfolgerung)
% Sie wird oft auch separat eingereicht.
% =============================================================================

\section*{Zusammenfassung}
\addcontentsline{toc}{section}{Zusammenfassung}

Kompakte \glspl{niederspannungsschaltanlage} erfordern konstruktionsbedingt geringe \glspl{phasenmittelabstand}. Die dabei auftretenden hohen Betriebsströme führen zu starken magnetischen \glspl{fremdfeld}, welche die Messgenauigkeit von \glspl{messstromwandler} signifikant beeinträchtigen können. Insbesondere die Störeinkopplung auf die mittlere Phase (\acrshort{l2}) stellt eine Herausforderung für die Einhaltung der \gls{genauigkeitsklasse} dar.

Das Ziel dieser Arbeit ist die systematische Analyse dieser \gls{fremdfeld}beeinflussung. Es wird eine Methodik entwickelt, um die physikalischen Ursachen der Messabweichungen zu quantifizieren und Lösungsansätze für die Konstruktionspraxis abzuleiten.

Hierfür wird ein zweigleisiger Ansatz verfolgt, der Simulation und Messtechnik kombiniert. Zunächst wird der Aufbau eines \gls{quasi-stationaer}en Simulationsmodells mittels der \acrshort{fem}-Software \gls{femm} vorgestellt. Diese Methode bildet die nichtlinearen Materialeigenschaften der Wandlerkerne, wie die \gls{bh-kennlinie} und die \gls{magnetische-saettigung}, detailliert ab. Zur Parametrierung und Steuerung der 50-Hz-Simulationen wird eine eigens entwickelte Weboberfläche beschrieben.

Parallel zur Simulation wird der messtechnische Aufbau zur Validierung der Modelle am \gls{erwaermungspruefstand} erläutert. Dieser Prüfstand ermöglicht die Einspeisung realer Hochströme in verschiedene \gls{sammelschiene}n-Anordnungen. Zur systematischen Erfassung der Genauigkeitsmessungen nach DIN EN 61869-2 wird ein maßgeschneidertes VBA-Messprotokoll-Tool konzipiert und implementiert.

