% =============================================================================
% ABKÜRZUNGSVERZEICHNIS UND GLOSSAR
% -----------------------------------------------------------------------------
% Reduzierte Version für Fachpublikum (6./7. Semester).
% Trivialbegriffe und Grundlagen wurden entfernt.
% =============================================================================

% --- ABKÜRZUNGEN DEFINIEREN ---

% (Leer, da Begriffe wie L1, L2, L3, VDE, EMV für die Zielgruppe vorausgesetzt werden.
% Falls spezifische Abkürzungen hinzukommen, hier einfügen.)


% =============================================================================
% --- FACHBEGRIFFE FÜRS GLOSSAR DEFINIEREN (ALPHABETISCH SORTIERT) ---
% =============================================================================

\newglossaryentry{erwaermungspruefstand}{
    name={Erwärmungsprüfstand},
    plural={Erwärmungsprüfstände},
    description={Hochstrom-Prüfanlage der Rolf Janssen GmbH. Dient in dieser Arbeit als Quelle für reale Hochströme (bis mehrere kA) bei Kleinspannung zur Simulation der Betriebsbedingungen.}
}

\newglossaryentry{femm}{
    name={FEMM},
    plural={FEMM},
    description={Finite Element Method Magnetics. Die verwendete Open-Source-Software zur Simulation der magnetischen Flussdichteverteilung im 2D-Raum.}
}

\newglossaryentry{kompensationswiderstand}{
    name={Kompensationswiderstand},
    plural={Kompensationswiderstände},
    description={Serieller Widerstand im Sekundärkreis, der eingesetzt wird, um die geringe Eingangsimpedanz moderner Digitalmessgeräte auszugleichen und die Nennbürde des Wandlers zu erreichen.}
}

\newglossaryentry{quasi-stationaer}{
    name={Quasi-stationäres Modell},
    description={Der in dieser Arbeit entwickelte Simulationsansatz, bei dem das dynamische 50-Hz-Verhalten durch eine automatisierte Sequenz statischer FEMM-Berechnungen diskreter Zeitpunkte angenähert wird.}
}

\newglossaryentry{unterbuerdung}{
    name={Unterbürdung},
    description={Betriebszustand, bei dem die tatsächliche sekundäre Lastimpedanz deutlich kleiner ist als die spezifizierte Nennbürde, was insbesondere bei älteren Wandlern an moderner Messtechnik zu Fehlern führen kann.}
}

\newglossaryentry{wandler-kurzschlussklemme}{
    name={Wandler-Kurzschlussklemme},
    plural={Wandler-Kurzschlussklemmen},
    description={Spezielle Anschlussklemme, die das gefahrlose Kurzschließen der Sekundärwicklung vor dem Auftrennen des Messkreises ermöglicht.}
}