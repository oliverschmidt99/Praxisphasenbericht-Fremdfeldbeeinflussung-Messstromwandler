% =============================================================================
% HAUPTDOKUMENT (main.tex)
% =============================================================================

% GEÄNDERT: 'footsepline=0.4pt' hinzugefügt
\documentclass[11pt, a4paper, headsepline=0.4pt, footsepline=0.4pt]{scrartcl}

% --- EINSTELLUNGEN LADEN ---
% =============================================================================
% PRÄAMBEL (praeambel.tex)
% -----------------------------------------------------------------------------
% Lädt alle Pakete und globalen Einstellungen.
% =============================================================================

% --- GRUNDEINSTELLUNGEN ---
\usepackage[utf8]{inputenc}
\usepackage[T1]{fontenc}
\usepackage[ngerman]{babel}
\usepackage{graphicx}
\usepackage{csquotes}

% --- LAYOUT-DATEI LADEN ---
% =============================================================================
% LAYOUT-EINSTELLUNGEN (layout.tex)
% =============================================================================

% --- SEITENRÄNDER UND GRÖSSE DER KOPF-/FUSSZEILE ---
\usepackage[
a4paper,
left=25mm,
right=25mm,
top=25mm,
bottom=25mm,
headheight=2cm,
footskip=15mm
]{geometry}

% --- KOPF- UND FUSSZEILE (scrlayer-scrpage) ---
\usepackage{scrlayer-scrpage}
\KOMAoptions{headsepline=0.4pt, footsepline=0.4pt} % Korrekte Aktivierung der Linien
\pagestyle{scrheadings}
\clearpairofpagestyles
\automark[section]{section}

% Kopfzeile befüllen
\ihead{\includegraphics[height=1.5cm,keepaspectratio]{03_Ressourcen/Logo/logo_rolf-janssen_2024.pdf}}
\ohead{{\normalfont\normalsize\praktikumstitel} \normalfont~|~\Large\autorenname}

% Fußzeile befüllen
\ifoot{\rightmark}
\ofoot{Seite \thepage\ von \ref{TotPages}} % \ref ist korrekt für totpages

% --- SCHRIFTARTEN ---
\setkomafont{pagehead}{\kopfzeilenSchrift}
\setkomafont{pagefoot}{\fusszeilenSchrift}
\setkomafont{section}{\sectionSchrift}
\setkomafont{subsection}{\subsectionSchrift}


% --- LITERATURVERZEICHNIS (BIBLATEX) ---
\usepackage[backend=biber, style=alphabetic, sorting=ynt]{biblatex}

% --- MATHEMATIK ---
\usepackage{amsmath, amssymb, amsthm, latexsym, derivative}

% --- GRAFIKEN & DIAGRAMME ---
\usepackage{epsfig, tikz, subcaption, caption, adjustbox, eso-pic}
\usepackage[european]{circuitikz}
\usepackage{siunitx}
\sisetup{locale = DE}

\usepackage{svg}
\usepackage{filecontents}

% --- TABELLEN ---
\usepackage{booktabs, multirow, colortbl}

% --- SONSTIGE PAKETE ---
\usepackage{enumitem}
\usepackage{lastpage}
\usepackage{makeidx}
\makeindex
\usepackage[colorlinks=true, linkcolor=blue, citecolor=green, urlcolor=magenta, hidelinks]{hyperref}
\usepackage{chngcntr}
\usepackage[normalem]{ulem}
\usepackage{calc}
\usepackage{pdfpages}
\usepackage{blindtext}
\usepackage{float}
\usepackage{newfloat}
\usepackage{xcolor}
\usepackage[most]{tcolorbox}
\usepackage[framemethod=TikZ]{mdframed}
\usepackage{etoolbox}

% --- BENUTZERDEFINIERTE EINSTELLUNGEN ---
\setlength{\parindent}{0em}
% GELÖSCHT: \setkomafont{sectioning}{\sffamily} - wird jetzt in layout.tex gemacht

% --- DEINE FLOAT-UMGEBUNGEN ---
\DeclareFloatingEnvironment[listname={Diagrammverzeichnis}, name=Diagramm, fileext=lod, placement=htp]{diagram}
\DeclareFloatingEnvironment[listname={Oszillogrammverzeichnis}, name=Oszillogramm, fileext=loo, placement=htp]{oszillo}

% --- ZÄHLER-EINSTELLUNGEN ---
\counterwithin{figure}{section}
\counterwithin{table}{section}
\counterwithin{oszillo}{section}
\counterwithin{diagram}{section}

% --- DEINE BOX-DEFINITIONEN ---
\newmdenv[linecolor=red, linewidth=1pt, frametitle=Hinweis, frametitlebackgroundcolor=red, frametitlefont=\color{white}\bfseries, frametitlerule=true, backgroundcolor=white, roundcorner=2pt]{hinweisbox}
\newmdenv[linecolor=gray, linewidth=1pt, frametitle=Aufgabe, frametitlebackgroundcolor=white, frametitlefont=\color{black}\bfseries, frametitlerule=true, backgroundcolor=white, roundcorner=2pt]{aufgabenbox}
\newcounter{rechnungbox}[section]
\renewcommand{\therechnungbox}{\arabic{rechnungbox}}
\newtcolorbox{rechnungbox}[1][]{colframe=darkgray, colback=white, coltitle=white, colbacktitle=gray, title={Rechnung \therechnungbox}, fonttitle=\bfseries, boxed title style={size=small}, boxrule=1pt, sharp corners=south, before upper=\stepcounter{rechnungbox}, #1}

% --- DEINE TOGGLES ---
\newtoggle{showtoc}
\newtoggle{showtables}
\newtoggle{showfigures}
\newtoggle{showoszillos}
\newtoggle{showdiagrams}
\toggletrue{showtoc}
\togglefalse{showtables}
\toggletrue{showfigures}
\togglefalse{showoszillos}
\togglefalse{showdiagrams}

% --- Globale Schriftart für den Haupttext aktivieren ---
\AtBeginDocument{\haupttextSchrift}
% =============================================================================
% EIGENE BEFEHLE & GLOBALE EINSTELLUNGEN (befehle.tex)
% -----------------------------------------------------------------------------
% Dies ist deine zentrale Steuerdatei für das gesamte Dokument.
% Hier passt du globale Variablen, Schriftarten, Abstände und
% Standardgrößen an.
% =============================================================================

% --- GLOBALE DEFINITIONEN FÜR DIE TITELSEITE ---
\newcommand{\praktikumstitel}{Praxisphasenbericht}
\newcommand{\semester}{}
\newcommand{\versuchsnummer}{}
\newcommand{\versuchstitel}{Fremdfeldbeeinflussung auf Messstromwandler in der Niederspannung}
\newcommand{\gruppe}{}
\newcommand{\studiengang}{Studiengang Elektrotechnik}

% --- Studentendefinitionen (wie von dir bereitgestellt) ---
\newcommand{\vornameStudEins}{}
\newcommand{\nachnameStudEins}{}
\newcommand{\matrikelnummerStudEins}{}
\newcommand{\vornameStudZwei}{Oliver}
\newcommand{\nachnameStudZwei}{Schmidt}
\newcommand{\matrikelnummerStudZwei}{7023462}
\newcommand{\autorenname}{\vornameStudZwei~\nachnameStudZwei} % Dieser Befehl ist gut!

% --- Betreuerdefinitionen (wie von dir bereitgestellt) ---
\newcommand{\betreuerEins}{Dr. Sandro Günter}
\newcommand{\betreuerZwei}{Dipl.-Ing. Holger Kuhlemann}
\newcommand{\betreuerDrei}{Dipl.-Ing. Rainer Ludewig}
\newcommand{\betreuerVier}{Simon Westerbur, B. Eng.}

% === NEU: Definitionen für Hochschule und Firma (für 00_deckblatt.tex) ===

% --- Hochschule (Basierend auf deinem Original-Deckblatt) ---
\newcommand{\hochschullogo}{03_Ressourcen/Logo/Technik.png}
\newcommand{\hochschulname}{Hochschule Emden/Leer}
\newcommand{\fachbereich}{Fachbereich Technik}
\newcommand{\abteilung}{Abteilung Elektrotechnik und Informatik}
\newcommand{\hochschuladresse}{Constantiaplatz 4 \\ 26723 Emden} % Beispiel-Adresse, kannst du anpassen

% --- Firma (Basierend auf deiner Eingabe) ---
\newcommand{\firmenlogo}{03_Ressourcen/Logo/logo_rolf-janssen_2024.pdf} % Pfad korrigiert
\newcommand{\firma}{Rolf Janssen GmbH Elektrotechnische Werke}
\newcommand{\firmenadresse}{Emsstraße 4 \\ 26603 Aurich} % Adresse habe ich für dich recherchiert

% =====================================================================


% --- HINTERGRUNDBILD ---
\newcommand\BackgroundWave{%
    \put(0,0){%
        \parbox[b][\paperheight]{\paperwidth}{%
            \vfill
            \centering
            \vspace{12.0cm}
            \includegraphics[width=\paperwidth]{03_Ressourcen/Logo/hsel-welle-grey}%
            \vfill
        }%
    }%
}


% #############################################################################
% ### ANFANG: GLOBALE LAYOUT- UND SCHRIFTART-EINSTELLUNGEN                ###
% #############################################################################
%
% Hier kannst du das gesamte Aussehen deines Dokuments anpassen,
% indem du einfach die Werte und Befehle änderst.
% === HAUPTSCHRIFTARTEN ===
% Um die Schriftart zu wechseln, lade das entsprechende Paket in 'praeambel.tex'
% und ändere hier den Befehl.
% Beispiele für Pakete: \usepackage{mathpazo} (Palatino), \usepackage{newtxtext} (Times)
\newcommand{\haupttextSchrift}{\rmfamily}       % \rmfamily (Serif), \sffamily (Sans-Serif)

% === SCHRIFTGRÖSSEN & -STILE ===
% Beispiele: \sffamily (serifenlos), \rmfamily (Serifen), \ttfamily (Schreibmaschine)
%           \bfseries (fett), \itshape (kursiv), \scshape (Kapitälchen)
%           \Huge, \huge, \LARGE, \Large, \large, \normalsize, \small, \footnotesize
\newcommand{\haupttextGroesse}{\normalsize}
\newcommand{\kapitelSchrift}{\sffamily\bfseries\Large}
\newcommand{\sectionSchrift}{\sffamily\bfseries\large}
\newcommand{\subsectionSchrift}{\sffamily\bfseries\normalsize}
\newcommand{\kopfzeilenSchrift}{\haupttextSchrift\small}
\newcommand{\fusszeilenSchrift}{\haupttextSchrift\footnotesize}

% === ABSTÄNDE ===
\newcommand{\zeilenabstand}{1.15}            % 1.0 für einfach, 1.15 für etwas luftiger, 1.5 für 1,5-fachen Abstand
\newcommand{\absatzabstand}{0.5em}             % Abstand zwischen Absätzen (z.B. 1em, 0.5cm, etc.)
\newcommand{\absatzeinzug}{0em}              % Einzug der ersten Zeile eines Absatzes (0em für linksbündig)

% === KOPF- UND FUSSZEILE ===
\newcommand{\kopfzeilenHoehe}{2.5cm}           % Reservierter Platz für die Kopfzeile
\newcommand{\fusszeilenAbstand}{1.5cm}           % Abstand der Fußzeile vom Text


% =============================================================================
% ### BEFEHLE FÜR AUTOMATISIERTE UMGEBUNGEN ###
% =============================================================================

% Befehl für eine kleine Abbildung (ca. 1/3 der Seitenbreite)
% #1 = Dateipfad
% #2 = Caption (Unterschrift)
% #3 = Label zur Referenzierung
\newcommand{\einfuegenBildKlein}[3]{%
    \begin{figure}[H]
        \centering
        \includegraphics[width=0.4\textwidth]{#1'Bild'}
        \caption{#2}
        \label{#3}
    \end{figure}%
}

% Befehl für eine mittlere Abbildung (ca. 2/3 der Seitenbreite)
% #1 = Dateipfad
% #2 = Caption (Unterschrift)
% #3 = Label zur Referenzierung
\newcommand{\einfuegenBildMittel}[3]{%
    \begin{figure}[H]
        \centering
        \includegraphics[width=0.6\textwidth]{#1}
        \caption{#2}
        \label{#3}
    \end{figure}%
}

% Befehl für eine große Abbildung (fast die ganze Seitenbreite)
% #1 = Dateipfad
% #2 = Caption (Unterschrift)
% #3 = Label zur Referenzierung
\newcommand{\einfuegenBildGross}[3]{%
    \begin{figure}[H]
        \centering
        \includegraphics[width=0.9\textwidth]{#1}
        \caption{#2}
        \label{#3}
    \end{figure}%
}

% #############################################################################
% ### ENDE: GLOBALE LAYOUT- UND SCHRIFTART-EINSTELLUNGEN                ###
% #############################################################################

% --- LITERATURQUELLE ANGEBEN ---
\addbibresource{bibliographie.bib}

% --- DEFINITIONEN FÜR GLOSSAR LADEN ---
% HINWEIS: Dieser Input-Befehl muss vor \begin{document} stehen!
% =============================================================================
% ABKÜRZUNGSVERZEICHNIS UND GLOSSAR
% -----------------------------------------------------------------------------
% Zentrale Definitionsdatei für alle Abkürzungen und Fachbegriffe.
% =============================================================================

% --- ABKÜRZUNGEN DEFINIEREN ---
\newacronym{l1}{L1}{Leiter 1 (erste Phase des Drehstromsystems)}
\newacronym{l2}{L2}{Leiter 2 (zweite Phase des Drehstromsystems)}
\newacronym{l3}{L3}{Leiter 3 (dritte Phase des Drehstromsystems)}
\newacronym{emv}{EMV}{Elektromagnetische Verträglichkeit}


% --- FACHBEGRIFFE FÜRS GLOSSAR DEFINIEREN ---
\newglossaryentry{sammelschiene}{
    name={Sammelschiene},
    description={Ein massiver elektrischer Leiter aus Kupfer oder Aluminium in einer Schaltanlage, der zur Verteilung hoher Ströme auf mehrere Abgänge dient.}
}

\newglossaryentry{fremdfeld}{
    name={Fremdfeld},
    description={Ein magnetisches oder elektrisches Feld, das von einer externen Quelle erzeugt wird und als Störgröße auf ein betrachtetes System einwirkt.}
}

\newglossaryentry{messstromwandler}{
    name={Messstromwandler},
    description={Ein spezieller Transformator, der hohe Primärströme in ein proportionales, kleines und für Messgeräte verarbeitbares Sekundärsignal umwandelt.}
}

% -----------------------------------------------------------------------------
% DOKUMENTENBEGINN
% -----------------------------------------------------------------------------
\begin{document}

% Deckblatt wird als unnummerierte Titelseite eingefügt
% =============================================================================
% DECKBLATT (00_deckblatt.tex)
% -----------------------------------------------------------------------------
% Dies ist die Titelseite deiner Arbeit. Sie verwendet die globalen
% Befehle aus 'befehle.tex', um die Informationen dynamisch einzufügen.
% =============================================================================

\begin{titlepage}

	\hspace{-1.0cm}
	\begin{tabular}{p{8.0cm} p{8.0cm}}
		% KORRIGIERT: Der Pfad zum Logo wurde an die neue Ordnerstruktur angepasst.
		\includegraphics[width = 6.0cm]{03_Ressourcen/Logo/Technik.png} & % Logo
		\parbox[b]{8.0cm}{
		{\large  Fachbereich Technik }                                    \\ % Fachbereich
		{\large  Abteilung Elektrotechnik und Informatik } % Abteilung
		}                                                                 \\
		\\
		\hline
	\end{tabular}
	%
	\begin{center}

		\vspace{2.5cm}
		% KORRIGIERT: Zeilenumbruch \\ entfernt, da \semester leer ist
		\LARGE{\textsc{\praktikumstitel}}
		\ifx\semester\empty\else\LARGE{\textsc{\semester}}\fi

		\vspace{2cm}%
		\ifx\versuchsnummer\empty\else\LARGE{\textsc{\versuchsnummer}}\fi
		\LARGE{\versuchstitel}

		\vspace{4cm}%
		\large

		% KORRIGIERT: Zeilenumbrüche entfernt, da \gruppe leer ist
		\ifx\gruppe\empty\else\gruppe\fi
		\studiengang
		\vspace{0.5cm}

		Vorgelegt von\\

		\begin{table}[!ht]
			\centering
			\begin{tabular}{rll}
				% KORRIGIERT: Leere erste Zeile entfernt
				\vornameStudZwei & \nachnameStudZwei & \matrikelnummerStudZwei % Studierender 2 (global definiert)
			\end{tabular}
		\end{table}


		\vspace{1cm}
		Emden, \today % Ort und Datum

		\vspace{1.5cm}%
		Betreut von\\ % Einleitung Betreuer
		\betreuerEins\\ % Betreuer 1 (global definiert)
		\betreuerZwei\\ % Betreuer 2 (global definiert)
		\betreuerDrei % Betreuer 3 (neu hinzugefügt)

	\end{center}
	\normalsize
\end{titlepage}

% --- VORSAPNN ---
\pagenumbering{Roman} % Römische Seitenzahlen für den Vorspann
% =============================================================================
% ABSTRACT (01_abstract.tex)
% -----------------------------------------------------------------------------
% Eine kurze, prägnante Zusammenfassung deiner gesamten Arbeit.
% (Thema, Vorgehen/Methode, wichtigste Ergebnisse, Schlussfolgerung)
% Sie wird oft auch separat eingereicht.
% =============================================================================

\section*{Abstract}
\addcontentsline{toc}{section}{Abstract}

Hier fasst du deine Arbeit kurz und prägnant zusammen.
% =============================================================================
% VERZEICHNISSE (02_verzeichnisse.tex)
% -----------------------------------------------------------------------------
% Diese Datei generiert alle Verzeichnisse (Inhalt, Abbildungen etc.)
% basierend auf den Schaltern (\toggletrue/\togglefalse) in der Präambel.
% =============================================================================

% Inhaltsverzeichnis
\iftoggle{showtoc}{
    \hypersetup{linkcolor=black}
    \tableofcontents
    \newpage
}{}

% Tabellenverzeichnis
\iftoggle{showtables}{
    \hypersetup{linkcolor=black}
    \listoftables
    \addcontentsline{toc}{section}{Tabellenverzeichnis}
    \newpage
}{}

% Abbildungsverzeichnis
\iftoggle{showfigures}{
    \listoffigures
    \addcontentsline{toc}{section}{Abbildungsverzeichnis}
    \newpage
}{}

% Oszillogrammverzeichnis
\iftoggle{showoszillos}{
    \listofoszillo
    \addcontentsline{toc}{section}{Oszillogrammverzeichnis}
    \newpage
}{}

% Diagrammverzeichnis
\iftoggle{showdiagrams}{
    \listofdiagram
    \addcontentsline{toc}{section}{Diagrammverzeichnis}
    \newpage
}{}

% Setzt die Linkfarbe für den Haupttext wieder auf Blau
\hypersetup{linkcolor=blue}

% --- GLOSSAR & ABKÜRZUNGSVERZEICHNIS AUSGEBEN ---
\printglossary[type=\acronymtype, title={Abkürzungsverzeichnis}]
\printglossary[title={Glossar}]
\newpage

% --- HAUPTTEIL ---
\pagenumbering{arabic} % Arabische Seitenzahlen für den Hauptteil
% =============================================================================
% KAPITEL 1: VORSTELLUNG DES UNTERNEHMENS
% -----------------------------------------------------------------------------
% Hier stellst du das Unternehmen vor, bei dem du die Praxisphase absolviert hast.
% =============================================================================

\section{Vorstellung des Unternehmens}\label{chap:vorstellung_unternehmen}

Das Unternehmen, die Rolf Janssen GmbH, wurde im Jahr 1949 gegründet und blickt auf eine mehr als 75-jährige Geschichte im Bereich der Elektrotechnik zurück.
Mit Hauptsitz in Aurich, hat sich das Unternehmen kontinuierlich entwickelt und beschäftigt heute über 200 Mitarbeiter an insgesamt fünf Standorten in Deutschland.
\einfuegenBildGross{03_Ressourcen/Bilder/abbildung_01_firmenzentrale.jpg} {Firmenzentrale der Rolf Janssen GmbH in Aurich} {fig:firmenzentrale}

Das Kerngeschäft der Rolf Janssen GmbH liegt in der Planung, Fertigung und Inbetriebnahme von elektrotechnischen und leittechnischen Anlagen.
Das Leistungsportfolio ist dabei breit diversifiziert im Bereich der Schaltanlagen und konzentriert sich auf die folgenden Hauptgeschäftsfelder:

\begin{itemize}
    \item \textbf{Niederspannungsschaltanlagen}: Die Entwicklung und Fertigung von maßgeschneiderten Schaltanlagen bilden eine zentrale Säule des Unternehmens.
Hierbei werden individuelle Kundenanforderungen präzise umgesetzt.
    \item \textbf{Prozessautomatisierung}: Ein weiterer Schwerpunkt ist die Konzeption und Implementierung von modernen Automatisierungslösungen für industrielle Fertigungs- und Verfahrensprozesse.
    \item \textbf{Diesel-Netzersatzanlagen}: Das Unternehmen stellt Notstromaggregate her, die eine unterbrechungsfreie Stromversorgung für kritische Infrastrukturen und industrielle Anwendungen gewährleisten.
    \item \textbf{Maritime Lösungen}: Für den anspruchsvollen maritimen Sektor werden spezialisierte Elektro- und Leittechniksysteme entwickelt, die den besonderen Anforderungen auf See (z. B. auf Schiffen oder Offshore-Plattformen) gerecht werden.
\end{itemize}

Ein wesentliches Alleinstellungsmerkmal der Rolf Janssen GmbH ist die hohe Fertigungstiefe.
Diese ermöglicht es, den gesamten Produktionsprozess – vom Rohmaterial bis zum betriebsbereiten Endprodukt – im eigenen Haus zu steuern.
Dadurch können Kundenanforderungen nicht nur flexibel, sondern auch kurzfristig umgesetzt werden.
Das Unternehmen legt besonderen Wert auf die Realisierung von anspruchsvollen Sonderlösungen und bietet eine durchgängige Projektbetreuung, die von der ersten Angebotsphase über die Konstruktion und Fertigung bis zur finalen Inbetriebnahme reicht.
Diese umfassende Herangehensweise positioniert die Rolf Janssen GmbH als kompetenten und zuverlässigen Partner für elektrotechnische Gesamtpakete in diversen Branchen.
% =============================================================================
% KAPITEL 2: PROJEKTBERICHT
% -----------------------------------------------------------------------------
% Das Kernstück deines Berichts. Hier beschreibst du dein Projekt.
% HINWEIS: Die \gls- und \acr-Befehle erzeugen automatisch die Links im PDF.
% =============================================================================

\section{Analyse der Fremdfeldbeeinflussung von Messstromwandlern}
\label{chap:projektbericht}

Im Rahmen dieser Praxisphase wird die Fremdfeldbeeinflussung auf Messstromwandler im Bereich der Niederspannung analysiert.
Ziel ist es, die physikalischen Ursachen für Messabweichungen, basierend auf dem Induktionsgesetz, zu untersuchen und Lösungsansätze zur Minimierung dieser Störeinflüsse zu entwickeln.
\subsection{Motivation}
\label{sec:motivation}

In modernen Niederspannungsschaltanlagen treffen konstruktive Anforderungen an eine kompakte Bauweise auf hohe Betriebsströme.
Die daraus resultierende kompakte Anordnung der Sammelschienen führt dazu, dass die starken magnetischen Wechselfelder der stromführenden Leiter auf benachbarte Messstromwandler koppeln.
Diese Fremdfeldbeeinflussung ist eine signifikante Ursache für Messfehler, die Einhaltung der Genauigkeitsklasse beeinträchtigen kann und somit für Verrechnungszwecke nicht mehr zulässig ist.
Besonders kritisch ist diese Kopplung für den Wandler der mittleren Phase L2, da dieser von den Feldern der beiden äußeren Phasen L1 und L3 beeinflusst wird.
\subsection{Vorgehen}
\label{sec:vorgehen}

Um die gegenseitige Beeinflussung der Magnetfelder zu untersuchen, wurde die Open-Source-Software \gls{femm} eingesetzt.
Mit dieser lassen sich verschiedene elektrotechnische Problemstellungen wie magnetische, elektrostatische, stromfluss oder thermische Phänomene simulieren.
Da die Software jedoch ausschließlich stationäre Probleme in zwei Dimensionen berechnen kann, wurde für die Analyse der Magnetfeldeinkopplung in die Messstromwandler eine ergänzende Software zur Simulation dynamischer Vorgänge entwickelt.
Die messtechnische Untersuchung der Messstromwandler findet im \gls{erwaermungspruefstand} statt. Sie gliedert sich in zwei Bereiche: Zuerst werden die internen Parameter mittels ACV-Messverfahren- und DCV-Messverfahren-Messungen ermittelt.
Anschließend erfolgt die Genauigkeitsmessung unter Betriebsbedingungen mit den jeweiligen Nennströmen der Wandler.
\subsubsection{Vorbereitung}
\label{sec:vorgehen:vorbereitung}

Als Vorbereitung für die messtechnischen und simulativen Untersuchungen wurden Messstromwandler verschiedener Hersteller mit unterschiedlichen Nennströmen beschafft.
Parallel dazu erfolgte eine umfassende Einarbeitung in die theoretischen Grundlagen der Messwandlertechnik und der Fremdfeldproblematik.
Dies umfasste die Recherche und Analyse relevanter Fachartikel und Bücher. Ergänzend wurden erste analytische Berechnungen zur Magnetfeldeinkopplung mittels MATLAB durchgeführt.
\subsection{Erwärmungsprüfstand}
\label{sec:erwaermungspruefstand}

Der \gls{erwaermungspruefstand} (siehe Abbildung \ref{fig:erwaermungspruefstand_ohne_wandler}) ist die zentrale Messeinrichtung für dieses Projekt.
Obwohl sein primärer Zweck, wie der Name andeutet, die thermische Analyse von Niederspannungsschaltanlagen nach VDE-Norm ist, liegt sein entscheidender Vorteil für diese Untersuchung in einer anderen Eigenschaft: Er kann die für den Realbetrieb typischen, sehr hohen Nennströme bis zu mehreren Kiloampere erzeugen.
Der Prüfstand nutzt einen Hochstromtransformator, um die benötigten Ströme bei sehr geringer Spannung (typischerweise < 10 V) in das Sammelschienen-System einzuspeisen.
Für die Analyse der Fremdfeldbeeinflussung ist dies die ideale Voraussetzung: Die hohen Ströme erzeugen die realitätsnahen, starken Magnetfelder, deren Kopplung auf die Messstromwandler untersucht werden soll.
Wie in Abschnitt \ref{sec:messung:messaufbau} detailliert beschrieben, wurde für die Messungen ein spezielles, modulares Kupferschienensystem im Prüfstand installiert.
Dieses System erlaubt es, die Messstromwandler schnell zu montieren und verschiedene geometrische Anordnungen (insbesondere unterschiedliche Phasenmittelabstände) zu realisieren, um deren Einfluss auf die Messgenauigkeit systematisch zu untersuchen.
\subsection{Simulation}
\label{sec:simulation}

Für die Simulation wurden verschiedene Ansätze verfolgt. Zunächst kam ein in MATLAB implementiertes Skript zum Einsatz, das die Magnetfelder auf Grundlage des Biot-Savart-Gesetzes berechnete.
Schnell zeigte sich jedoch, dass dieser Ansatz die physikalische Realität nur unzureichend abbildete. Daher wurde eine alternative Methode erforderlich.
Nach einer ausführlichen Recherche geeigneter, kostenfreier Open-Source-Software für Magnetfeldsimulationen fiel die Wahl auf \gls{femm} (Finite Element Method Magnetics).
\Gls{femm} ist ein Programm, das magnetische Probleme mithilfe der FEM löst.
Der entscheidende Vorteil gegenüber dem Biot-Savart-Ansatz liegt darin, dass \gls{femm} nicht nur komplexe Leitergeometrien, sondern auch die nichtlinearen Materialeigenschaften ferromagnetischer Werkstoffe (z. B. B(H)-Kennlinie und magnetische Sättigung) berücksichtigt.
Genau diese Eigenschaften wurden beim reinen Biot-Savart-Modell vernachlässigt, was zu deutlichen Abweichungen führte.
Daher wurde das Modell in \gls{femm} neu aufgebaut, um die Sättigungseffekte realitätsnah abzubilden.
\subsubsection{Maßgeschneiderte Simulationsumgebung}
\label{sec:simulation:software}

Für die spezifischen Anforderungen des Messaufbaus wurde auf Basis von \gls{femm} eine maßgeschneiderte Softwarelösung entwickelt, die die Fremdfeldbeeinflussung gezielt abbildet.
Diese Lösung stellt eine Weboberfläche (siehe Abbildung \ref{fig:weboberflaeche}) bereit, über die sich die Simulationen definieren, parametrieren und starten lassen (siehe Abbildung \ref{fig:weboberflaeche_einstellungen}).
Die Oberfläche dient als einheitliche Vorlage (Template) und ermöglicht die standardisierte Erstellung von Simulationsmodellen für verschiedene Wandlertypen und Sammelschienenanordnungen.
Da \Gls{femm} ausschließlich stationäre Fälle betrachtet, implementiert die Hilfssoftware ein \gls{quasi-stationaer}, um die dynamischen Vorgänge anzunähern.
Zu diskreten Zeitpunkten $t_n$ innerhalb einer Netzperiode berechnet die Hilfssoftware die jeweiligen Momentanwerte der Phasenströme $i_{L1}(t_n)$, $i_{L2}(t_n)$ und $i_{L3}(t_n)$.
Diese Werte werden automatisiert an \gls{femm} übergeben, das für jeden Zeitschritt ein stationäres Magnetfeldproblem löst.
Durch die Aneinanderreihung der Einzelergebnisse – insbesondere des magnetischen Flusses $\Psi(t_n)$ im Wandlerkern – lässt sich der zeitliche Verlauf über eine volle Periode rekonstruieren.
\einfuegenBildGross{03_Ressourcen/Bilder/website_Simulations-Vorschau.png}{Weboberfläche der Simulationssoftware mit Vorschau des Modells}{fig:weboberflaeche}

\einfuegenBildGross{03_Ressourcen/Bilder/website_Simulations-einstellungen.png}{Weboberfläche der Simulationssoftware mit Einstellmöglichkeiten für die Simulation}{fig:weboberflaeche_einstellungen}


\subsubsection{Detaillierter Modellaufbau}
\label{sec:simulation:modellaufbau}

Der detaillierte Modellaufbau in \gls{femm} erfolgte in einem zweidimensionalen Querschnitt, der die Anordnung der Sammelschienen und des Messstromwandlers abbildet:

\begin{itemize}
    \item Geometrie: Die drei Phasenleiter wurden entsprechend ihrer realen Querschnittsabmessungen (z. B. 100 mm × 10 mm) und ihres Mittenabstands modelliert.
Ebenso wurde der Ringkern des Messstromwandlers mit seinem exakten Innen- und Außendurchmesser sowie seiner Position relativ zu den Leitern nachgebildet.
    \item Materialien: Den Leitern wurde Kupfer mit seiner spezifischen elektrischen Leitfähigkeit zugewiesen, der Umgebungsraum als Luft ($\mu_r = 1$) definiert.
Der ferromagnetische Wandlerkern erhielt eine nichtlineare B(H)-Kennlinie aus der FEMM-Materialbibliothek (z. B. M-19 Steel), um das Sättigungsverhalten realitätsnah zu erfassen.
\end{itemize}

\subsection{Messung}
\label{sec:messung}

Bei der Messung am \gls{erwaermungspruefstand} sollen die Messstromwandler in verschiedenen geometrischen Anordnungen auf ihre Genauigkeit hin überprüft werden.
Die Anordnung der Kupferschienen wurde für diese Messungen speziell angefertigt, wodurch sich die Messstromwandler möglichst einfach und schnell in den verschiedenen Positionen vermessen lassen.
\subsubsection{Messaufbau}
\label{sec:messung:messaufbau}

Ziel der Messungen ist es zu überprüfen, ob die Messstromwandler durch geometrische Anpassung weiterhin unter dem Einfluss von Fremdfeldern, die Anforderungen an die Genauigkeitsklasse der DIN-Norm 61869-2 einhalten.
Die Grenzwerte für die Übersetzungsmessabweichung sind dazu in Tabelle~\ref{tab:grenzwerte_wandler_reduziert} dargestellt.
\begin{table}[H]
    \centering
    \caption{Grenzwerte für Übersetzungsmessabweichung für Stromwandler für Messzwecke (Klassen 0,1 bis 1) nach DIN EN 61869-2}
    \label{tab:grenzwerte_wandler_reduziert}
    \begin{tabular}{lcccc}
        \toprule
        \multirow{3}{*}{Genauigkeitsklasse} & \multicolumn{4}{c}{Übersetzungsmessabweichung}                                     \\
                                             & \multicolumn{4}{c}{$\pm \%$ bei Strom (\% von Bemessungsstrom)}                    \\
        \cmidrule(lr){2-5}
                                             & 5                                                               & 20   & 100 & 120 \\
        \midrule
        0,1                                  & 0,4                                                             & 0,2  & 0,1 & 0,1 \\
        0,2                                  & 0,75                                                            & 0,35 & 0,2 & 0,2 \\
        0,5                                  & 1,5                                                             & 0,75 & 0,5 & 0,5 \\
        1                                   & 3,0                                                              & 1,5  & 1,0 & 1,0 \\
        \bottomrule
    \end{tabular}
\end{table}

Zu diesem Zweck zeigt Abbildung \ref{fig:erwaermungspruefstand_ohne_wandler} das spezialisierte Kupferschienensystem des \gls{erwaermungspruefstand}.
Dieser Aufbau ermöglicht die Simulation von zwei verschiedenen Phasenmittelabständen. Der Phasenmittelabstand bezeichnet den Abstand zwischen den Außenleitern.
Dieses Maß wird durch den Hauptschalter bestimmt, an dem die Kupferschienen befestigt sind (im \gls{erwaermungspruefstand} ist kein Hauptschalter verbaut).
Das abgebildete System ist zweigeteilt:
\begin{itemize}
    \item Links im Bild ist das Schienensystem für Ströme von 630 A bis 1250 A mit einem Phasenmittelabstand von 70 mm zu sehen.
    \item Rechts im Bild befindet sich das System für Ströme von 630 A bis 3200 A mit einem Phasenmittelabstand von 90 mm.
\end{itemize}
Ein drittes Schienensystem (für 3200 A bis 5000 A mit 210 mm Phasenmittelabstand) ist nicht dargestellt, da es gesondert verbaut werden muss.
Um die Messstromwandler auf dem Schienensystem zu installieren, muss die Kurzschlussbrücke entfernt werden.
Anschließend können die Wandler von unten auf die Schienen der jeweiligen Phasen geschoben werden.
\einfuegenBildMittel{03_Ressourcen/Bilder/erwaermungspruefstand_abgang_ohne_wandler.png}{Messaufbau des Erwärmungsprüfstands ohne Messstromwandler}{fig:erwaermungspruefstand_ohne_wandler}

In der folgenden Abbildung \ref{fig:erwaermungspruefstand_wandler_pos2} ist die Position 2 realisiert.
Bei dieser Anordnung wird die Phase L2 durch die Installation einer zusätzlichen Kupferschiene mit einem Winkel nach vorne verlagert.
\einfuegenBildGross{03_Ressourcen/Bilder/erwaermungspruefstand_abgang_mit_wandler_pos2.png}{Messaufbau mit installierten Messstromwandlern in Position 2 (Phase L2 nach vorne verlagert)}{fig:erwaermungspruefstand_wandler_pos2}

Nach der Montage am Schienensystem werden die Messstromwandler verdrahtet.
Dafür werden die Anschlüsse der Sekundärwicklung auf spezielle \glspl{wandler-kurzschlussklemme} geführt.
Diese Klemmen ermöglichen ein sicheres Kurzschließen der Sekundärseite, bevor ein Messgerät entfernt oder getauscht wird.
Zudem bieten diese Klemmen die Möglichkeit, Messgeräte direkt über Standard-Labormessleitungen anzuschließen.
Dieser Schritt ist aus Sicherheitsgründen essenziell: Ein Messstromwandler darf niemals auf der Sekundärseite offen betrieben werden.
Im Leerlauf würden an den offenen Klemmen extrem hohe und gefährliche Spannungen entstehen.
Diese lebensgefährlichen Spannungsspitzen können die Isolation des Wandlers zerstören und stellen eine ernste Gefahr durch Stromschlag dar.
Ein weiterer wichtiger Aspekt ist die Bürde des Wandlers. Messstromwandler besitzen eine definierte Nennbürde (meist in VA angegeben), die im Betrieb eingehalten werden muss, damit der Wandler korrekte Messergebnisse innerhalb seiner Genauigkeitsklasse liefert.
Das Ersatzschaltbild in Abbildung \ref{fig:esb_buerde} visualisiert die Komponenten der sekundärseitigen Gesamtimpedanz, die sich aus den Leitungswiderständen ($R_{Kabel}$), dem Messgerät ($A$) und dem, falls nötig, zusätzlichen \gls{kompensationswiderstand} ($R_{Kompensation}$) zusammensetzt.
\einfuegenBildGross{03_Ressourcen/Bilder/ESB_Messwandler.pdf}{Ersatzschaltbild der sekundärseitigen Bürde (Kabel, Messgerät und ggf. Kompensation)}{fig:esb_buerde}

Früher waren analoge Dreheisenmesswerke der Standard zur Strommessung.
Da diese Messwerke eine signifikante interne Bürde (Eigenverbrauch) aufweisen, wurden die Wandler entsprechend für eine höhere Ausgangsbürde ausgelegt.
Heutzutage erfolgen Messungen primär mittels digitaler Messgeräte. Diese sind im Strommessbereich extrem niederohmig (die Impedanz liegt oft deutlich unter 1 $\Omega$).
Wird ein älterer Wandler, der für eine hohe Bürde ausgelegt ist, mit einem modernen, niederohmigen Messgerät betrieben, wird die Nennbürde nicht mehr erreicht.
Man spricht von einer \gls{unterbuerdung}. Dieser Betriebszustand führt zu Messfehlern.
Um dies zu kompensieren und den Wandler in seinem korrekten Arbeitsbereich zu betreiben, müssen zusätzliche \glspl{kompensationswiderstand} in den Sekundärkreis geschaltet werden (siehe in Abbildung \ref{fig:esb_buerde}), um die Gesamtimpedanz auf die Nennbürde des Wandlers zu erhöhen.
\subsubsection{Messprotokoll}
\label{sec:messung:messprotokoll}

Zur systematischen Erfassung und späteren Auswertung der Messdaten ist ein strukturiertes Messprotokoll erforderlich.
Die Herausforderung besteht in der repetitiven Art der Genauigkeitsmessung: Für eine Vielzahl unterschiedlicher Messstromwandler mit verschiedenen Nennströmen müssen die identischen Messprozeduren (siehe Tabelle \ref{tab:grenzwerte_wandler_reduziert}) durchgeführt und protokolliert werden.
Ein erster Ansatz bestand darin, eine standardisierte Tabellenvorlage in Excel zu erstellen.
Diese Vorlage wurde anschließend für jeden zu messenden Wandler manuell kopiert.
Diese Methode erwies sich jedoch schnell als unpraktikabel und fehleranfällig.
Die wesentlichen Nachteile waren:

\begin{itemize}
    \item Mangelnde Wartbarkeit: Nachträgliche Änderungen oder Erweiterungen an der Vorlage (z.B. eine zusätzliche Berechnungsspalte) mussten mühselig in jeder einzelnen, bereits kopierten Datei manuell nachgepflegt werden.
    \item Schlechte Datenvergleichbarkeit: Ein direkter Vergleich der Messwerte verschiedener Wandler war nicht ohne Umwege (z.B. erneutes manuelles Zusammenkopieren) möglich, da die Daten dezentral in separaten Dateien vorlagen.
\end{itemize}

Aus diesen Gründen wurde ein zentralisiertes Messprotokoll-Tool in Excel konzipiert und mittels VBA (Visual Basic for Applications) umgesetzt.
Dieses Tool trennt die Dateneingabe klar von der Datenspeicherung:

\begin{itemize}
    \item Zentrales Dashboard: Eine grafische Oberfläche (Dashboard) dient als zentrale Steuer- und Eingabemaske.
Hier kann der zu prüfende Messstromwandler ausgewählt und neue Messwerte können erfasst werden.
    \item Datenbank-Struktur: Die Stammdaten der Wandler (z.B. Typ, Hersteller, Nennstrom) sowie die erfassten Messwerte werden in separaten, als Datenbank fungierenden Tabellenblättern gespeichert.
\end{itemize}

Dieser Aufbau stellt die Datenkonsistenz sicher, vereinfacht die Wartung des Tools und ermöglicht jederzeit eine globale, vergleichbare Auswertung aller erfassten Messdaten.
\einfuegenBildGross{03_Ressourcen/Bilder/tab_dashboard.png}{Dashboard des VBA-Messprotokolls}{fig:tabelle_dashboard}

Das in Abbildung \ref{fig:tabelle_dashboard} gezeigte Dashboard dient als zentrale Startseite.
Hier wird zu Beginn der Messung der Messstromwandler ausgewählt, der geprüft werden soll.
Dessen Stammdaten – also Hersteller, Nennstrom, Nennbürde und Genauigkeitsklasse – lädt man direkt aus der Datenbank.
Nach Abschluss aller Messungen und Berechnungen wird das Dashboard ebenfalls genutzt, um die bearbeiteten Daten wieder zu speichern.
\einfuegenBildGross{03_Ressourcen/Bilder/tab_buerde.png}{Modul zur Bürdenberechnung}{fig:tabelle_buerde}

Das Modul zur Bürdenberechnung (Abbildung \ref{fig:tabelle_buerde}) ist entscheidend für die korrekte Konfiguration des Messaufbaus, damit die Nennbürde des Wandlers exakt getroffen wird.
Man trägt hier die relevanten Parameter des Aufbaus ein: die Kabellänge zum Messgerät, den Leiterquerschnitt und den Innenwiderstand des Messgeräts.
Das Tool berechnet daraus die resultierende Kabel- und Messgerätebürde. Vor allem ermittelt es automatisch den notwendigen \gls{kompensationswiderstand} ($R_{komp}$), der eingesetzt werden muss, um eine \gls{unterbuerdung} zu verhindern und die Nennbürde präzise einzustellen.
\einfuegenBildGross{03_Ressourcen/Bilder/tab_magnetisierungskennline.png}{Modul zur Erfassung der Magnetisierungskennlinie}{fig:tabelle_magnetisierungskennliene}

Das in Abbildung \ref{fig:tabelle_magnetisierungskennliene} dargestellte Modul dient der Erfassung des Sättigungsverhaltens des Wandlerkerns.
Hier trägt man die gemessenen Wertepaare aus Spannung (U) und Magnetisierungsstrom (I) in die Tabelle ein.
Zusätzlich werden die geometrischen Kerndaten (z.B. Radius, Höhe, Breite) erfasst, die das Tool für die Auswertung benötigt.
Basierend auf den eingegebenen Daten wird die resultierende U(I)-Kennlinie direkt als Grafik dargestellt, was eine sofortige visuelle Beurteilung des Sättigungsverhaltens erlaubt.
Die Tabelle unterstützt die Analyse zusätzlich, indem sie automatisch dabei hilft, den Kniepunkt der Kennlinie zu identifizieren.
\einfuegenBildGross{03_Ressourcen/Bilder/tab_genauigkeit.png}{Modul zur Genauigkeitsprüfung}{fig:tabelle_genauigkeit}

Abbildung \ref{fig:tabelle_genauigkeit} zeigt das Kernmodul zur Genauigkeitsprüfung. Es dient der Eingabe und automatisierten Auswertung der Messwerte.
Nach Eingabe der Messwerte für die normativen Prüfpunkte (z.B. 5~\%, 20~\%, 100~\% $I_N$) der Phasen berechnet das Tool den Übersetzungsfehler ($\epsilon$).
Die Grafik visualisiert sofort, ob die Messkurve (blau) innerhalb der Grenzlinien (rot) liegt und der Wandler somit die Anforderungen seiner Genauigkeitsklasse erfüllt.
Die Vorgehensweise zur Ermittlung dieser Messwerte ist in Abschnitt \ref{sec:messung:messdurchfuehrung} beschrieben.
\subsubsection{Messdurchführung}
\label{sec:messung:messdurchfuehrung}

Die Messdurchführung am Prüfstand gliedert sich in zwei Hauptbereiche: die Ermittlung der Wandler-Basiskennwerte (B(H)-Kennlinie, Wicklungswiderstand) und die eigentliche Genauigkeitsmessung unter Last und Fremdfeldeinfluss.
\paragraph{Kennlinien- und Widerstandsmessung}
Zuerst werden die internen Parameter des Wandlers bestimmt.
Für die Aufnahme der B(H)-Kennlinie muss die Primärseite des Wandlers unbeschaltet, also im Leerlauf, sein.
Die Messung erfolgt sekundärseitig durch das Anlegen einer variablen AC-Spannung, wie in Abbildung \ref{fig:esb_messung_magnetisierungskennlinie} dargestellt.
Als Spannungsquelle diente eine regelbare AC-Quelle (Typ: Kocos Artis 600).
\einfuegenBildGross{03_Ressourcen/Bilder/esb_messung_magnetisierungskennlinie.pdf}{Messschaltung zur Aufnahme der Magnetisierungskennlinie}{fig:esb_messung_magnetisierungskennlinie}

Parallel zur Sekundärwicklung wird ein hochohmiger Widerstand geschaltet.
Dieser Schutzwiderstand begrenzt die beim Ummagnetisieren des Kerns entstehenden Stromspitzen und schützt so die angeschlossenen Messgeräte.
Bei der Messdurchführung wird die Spannung schrittweise von einem niedrigen Wert beginnend erhöht, bis die magnetische Sättigung des Kerns deutlich eintritt.
Um den Verlauf der Kennlinie im Bereich des Kniepunkts präzise abzubilden, wurden die Spannungsschritte in dieser Region bewusst verkleinert.
Zusätzlich wurde der Widerstand der Sekundärwicklung ($R_{CT}$) bestimmt. Hierfür wurde das Prüfgerät (Kocos Artis 600) als Gleichspannungsquelle konfiguriert.
Aus den gemessenen Werten einer stabilen Gleichspannung und des resultierenden Gleichstroms wurde der Widerstand nach dem Ohm'schen Gesetz berechnet.
\paragraph{Genauigkeitsmessung}
Nach der Ermittlung der Basiskennwerte erfolgt die zentrale Genauigkeitsmessung am \gls{erwaermungspruefstand}.
Dazu wird der Messstromwandler zunächst in der ersten geometrischen Anordnung (z.B. Position 1) auf dem Kupferschienensystem montiert.
Anschließend werden nacheinander die in der Norm (Tabelle \ref{tab:grenzwerte_wandler_reduziert}) geforderten Prüfströme (z.B. 5~\%, 20~\%, 100~\% und 120~\% des Nennstroms) auf die Primärseite aufgeprägt.
Für jeden einzelnen Prüfpunkt und für jede Phase (L1, L2, L3) werden die exakten Ströme der Primär- und Sekundärseite mithilfe eines Referenzmessgeräts erfasst.
Diese ermittelten Messwertpaare werden in das VBA-Messprotokoll (siehe Abbildung \ref{fig:tabelle_genauigkeit}) übertragen, welches daraus den Fehler berechnet.
Um den Einfluss der Fremdfelder durch die geometrische Anordnung zu untersuchen, wird der Messaufbau anschließend umgebaut.
Beispielsweise wird, wie in Abbildung \ref{fig:erwaermungspruefstand_wandler_pos2} gezeigt, die Anordnung für Position 2 realisiert, bei der die mittlere Phase L2 nach vorne verlagert ist.
Nach dem Umbau wird der gesamte Genauigkeits-Messzyklus für diese Position wiederholt.
Dieser Prozess wird für alle zu untersuchenden Wandler und geometrischen Anordnungen iteriert.
% =============================================================================
% KAPITEL 3: REFLEXION
% -----------------------------------------------------------------------------
% Hier reflektierst du über deine Erfahrungen während der Praxisphase.
% =============================================================================

\section{Reflexion}
\label{chap:reflexion}

\subsection{Was habe ich gelernt?}
\label{sec:gelernt}
Welche neuen fachlichen und persönlichen Kompetenzen konntest du erwerben?

\subsection{Was lief nicht so gut?}
\label{sec:schwierigkeiten}
Wo gab es Herausforderungen oder unerwartete Probleme?

\subsection{Wie hätte man es besser machen können?}
\label{sec:verbesserung}
Was würdest du im Nachhinein anders machen?
% =============================================================================
% KAPITEL 4: ZUSAMMENFASSUNG UND AUSBLICK
% -----------------------------------------------------------------------------
% Hier fasst du die wichtigsten Punkte zusammen und gibst einen Ausblick.
% =============================================================================

\section{Zusammenfassung und Ausblick}
\label{chap:zusammenfassung_ausblick}

Hier fasst du die zentralen Erkenntnisse deiner Arbeit zusammen und gibst einen Ausblick auf mögliche weiterführende Schritte.


\textcolor{red}{(Damit kein warning angezigt wird}

% ---   ---
\appendix
\phantomsection
\addcontentsline{toc}{section}{Anhang}
\section*{Anhang} % \section{Anhang} ist in der Datei selbst

% --- NACHSPANN ---
\printbibliography[heading=bibintoc, title={Literaturverzeichnis}]
% =============================================================================
% EIGENSTÄNDIGKEITSERKLÄRUNG (eigenstaendigkeitserklaerung.tex)
% -----------------------------------------------------------------------------
% Die formale, unterschriebene Versicherung, dass du die Arbeit
% selbstständig verfasst hast.
% =============================================================================
\addsec*{Eidesstattliche Versicherung}

Ich, der Unterzeichnende, erkläre hiermit an Eides statt, dass ich die vorliegende Arbeit
selbständig verfasst habe und keine anderen als die angegebenen Quellen und Hilfsmittel benutzt habe.
Alle Quellenangaben und Zitate sind richtig und vollständig wiedergegeben und in den
jeweiligen Kapiteln und im Literaturverzeichnis wiedergegeben.

Ferner erkläre ich, dass zur sprachlichen Verbesserung und Korrektur von Textpassagen künstliche Intelligenz (KI) unterstützend eingesetzt wurde. Die inhaltliche Erstellung der Arbeit erfolgte eigenständig.

Die vorliegende Arbeit wurde nicht in
dieser oder einer ähnlichen Form ganz oder in Teilen zur Erlangung eines akademischen
Abschlussgrades oder einer anderen Prüfungsleistung eingereicht.\\
Mir ist bekannt, dass falsche Angaben im Zusammenhang mit dieser Erklärung strafrechtlich verfolgt
werden können.

\vspace{3cm}

\noindent\makebox[5cm][l]{Ort, Datum} \hspace{4cm} \makebox[5cm][r]{Unterschrift}
\hrule

\end{document}